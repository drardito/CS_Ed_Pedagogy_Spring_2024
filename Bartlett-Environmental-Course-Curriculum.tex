% Options for packages loaded elsewhere
\PassOptionsToPackage{unicode}{hyperref}
\PassOptionsToPackage{hyphens}{url}
%
\documentclass[
]{article}
\usepackage{amsmath,amssymb}
\usepackage{lmodern}
\usepackage{iftex}
\ifPDFTeX
  \usepackage[T1]{fontenc}
  \usepackage[utf8]{inputenc}
  \usepackage{textcomp} % provide euro and other symbols
\else % if luatex or xetex
  \usepackage{unicode-math}
  \defaultfontfeatures{Scale=MatchLowercase}
  \defaultfontfeatures[\rmfamily]{Ligatures=TeX,Scale=1}
\fi
% Use upquote if available, for straight quotes in verbatim environments
\IfFileExists{upquote.sty}{\usepackage{upquote}}{}
\IfFileExists{microtype.sty}{% use microtype if available
  \usepackage[]{microtype}
  \UseMicrotypeSet[protrusion]{basicmath} % disable protrusion for tt fonts
}{}
\makeatletter
\@ifundefined{KOMAClassName}{% if non-KOMA class
  \IfFileExists{parskip.sty}{%
    \usepackage{parskip}
  }{% else
    \setlength{\parindent}{0pt}
    \setlength{\parskip}{6pt plus 2pt minus 1pt}}
}{% if KOMA class
  \KOMAoptions{parskip=half}}
\makeatother
\usepackage{xcolor}
\IfFileExists{xurl.sty}{\usepackage{xurl}}{} % add URL line breaks if available
\IfFileExists{bookmark.sty}{\usepackage{bookmark}}{\usepackage{hyperref}}
\hypersetup{
  hidelinks,
  pdfcreator={LaTeX via pandoc}}
\urlstyle{same} % disable monospaced font for URLs
\usepackage[margin=1in]{geometry}
\usepackage{longtable,booktabs,array}
\usepackage{calc} % for calculating minipage widths
% Correct order of tables after \paragraph or \subparagraph
\usepackage{etoolbox}
\makeatletter
\patchcmd\longtable{\par}{\if@noskipsec\mbox{}\fi\par}{}{}
\makeatother
% Allow footnotes in longtable head/foot
\IfFileExists{footnotehyper.sty}{\usepackage{footnotehyper}}{\usepackage{footnote}}
\makesavenoteenv{longtable}
\usepackage{graphicx}
\makeatletter
\def\maxwidth{\ifdim\Gin@nat@width>\linewidth\linewidth\else\Gin@nat@width\fi}
\def\maxheight{\ifdim\Gin@nat@height>\textheight\textheight\else\Gin@nat@height\fi}
\makeatother
% Scale images if necessary, so that they will not overflow the page
% margins by default, and it is still possible to overwrite the defaults
% using explicit options in \includegraphics[width, height, ...]{}
\setkeys{Gin}{width=\maxwidth,height=\maxheight,keepaspectratio}
% Set default figure placement to htbp
\makeatletter
\def\fps@figure{htbp}
\makeatother
\setlength{\emergencystretch}{3em} % prevent overfull lines
\providecommand{\tightlist}{%
  \setlength{\itemsep}{0pt}\setlength{\parskip}{0pt}}
\setcounter{secnumdepth}{-\maxdimen} % remove section numbering
\ifLuaTeX
  \usepackage{selnolig}  % disable illegal ligatures
\fi

\author{}
\date{\vspace{-2.5em}}

\begin{document}

\hypertarget{bartlett-environmental-course-curriculum}{%
\section{\texorpdfstring{\textbf{Bartlett Environmental Course
Curriculum}}{Bartlett Environmental Course Curriculum}}\label{bartlett-environmental-course-curriculum}}

{[}TOC{]}

\hypertarget{introduction}{%
\subsection{Introduction}\label{introduction}}

This brief introduction to this Environment Science course does several
things: \emph{describing} the goals for the course; \emph{discussing}
the design principles employed; \emph{outlining} the Essential Questions
and Big Ideas/Enduring Understandings; and \emph{outlining} the
Massachusetts learning standards addressed.

\hypertarget{goal-of-the-course}{%
\subsubsection{Goal of the Course}\label{goal-of-the-course}}

The goal of this Environment Science course is threefold:

\begin{enumerate}
\def\labelenumi{\arabic{enumi}.}
\tightlist
\item
  \textbf{Enhance} the science experience of incoming freshmen.
\item
  Allow students to \textbf{explore} the key concepts of the course
  through student-centered pedagogical strategies, such as:
  \href{https://www.edutopia.org/project-based-learning}{project-based
  learning},
  \href{https://www.ideo.com/post/design-thinking-for-educators}{design
  thinking}, and
  \href{https://www.edutopia.org/blog/what-heck-inquiry-based-learning-heather-wolpert-gawron}{inquiry-based
  learning}.
\item
  Provide students with a \textbf{rich experience} of environmental
  science, sustainability, and human impact on the environment.
\end{enumerate}

\hypertarget{essential-questions-and-big-ideasenduring-understandings}{%
\subsubsection{Essential Questions and Big Ideas/Enduring
Understandings}\label{essential-questions-and-big-ideasenduring-understandings}}

\hypertarget{essential-questions-for-the-course}{%
\paragraph{Essential Questions for the
Course}\label{essential-questions-for-the-course}}

\begin{enumerate}
\def\labelenumi{\arabic{enumi}.}
\item
  Living and non-things things are connected via
  \href{https://en.wikipedia.org/wiki/Complex_adaptive_system}{\textbf{complex,
  adaptive systems}}. When you touch one thing, you touch everything.
\item
  Living systems tend toward
  \href{https://en.wikipedia.org/wiki/Homeostasis}{\textbf{homeostasis}},
  however humans have an outsized impact on the environment which
  threatens this homeostasis.
\item
  People can learn to come together to design solutions that can
  positively influence the
  \href{https://www.epa.gov/sustainability/learn-about-sustainability}{\textbf{sustainability}}
  of our world.
\end{enumerate}

\hypertarget{big-ideasenduring-understandings-for-the-course}{%
\paragraph{Big Ideas/Enduring Understandings for the
Course}\label{big-ideasenduring-understandings-for-the-course}}

\begin{enumerate}
\def\labelenumi{\arabic{enumi}.}
\item
  How do living and non-living systems work and work together and
  respond to one another?
\item
  What is the value of diversity in terms of living and non-living
  systems, and how can this diversity be expanded?
\item
  In what ways have we had negative impacts on our world and how can we
  repair the damage we have caused?
\end{enumerate}

\hypertarget{learning-standards-addressed}{%
\subsubsection{Learning Standards
Addressed}\label{learning-standards-addressed}}

\includegraphics{https://images.unsplash.com/photo-1532094349884-543bc11b234d?ixlib=rb-1.2.1\&ixid=MnwxMjA3fDB8MHxwaG90by1wYWdlfHx8fGVufDB8fHx8\&auto=format\&fit=crop\&w=1170\&q=80}

This curriculum has been designed to address the Massachusetts
\href{https://www.doe.mass.edu/frameworks/scitech/2016-04.pdf}{2016
Science and Technology Engineering Framework}.

\begin{longtable}[]{@{}
  >{\raggedright\arraybackslash}p{(\columnwidth - 2\tabcolsep) * \real{0.5000}}
  >{\raggedright\arraybackslash}p{(\columnwidth - 2\tabcolsep) * \real{0.5000}}@{}}
\toprule
\begin{minipage}[b]{\linewidth}\raggedright
Standard
\end{minipage} & \begin{minipage}[b]{\linewidth}\raggedright
9-12 Description
\end{minipage} \\
\midrule
\endhead
ESS.1.5 Earth's Place in the Universe & Evaluate evidence of the past
and current movements of continental and oceanic crust, the theory of
plate tectonics, and relative densities of oceanic and continental rocks
to explain why continental rocks are generally much older than rocks of
the ocean floor. \\
ESS2.A Earth materials and systems & Feedback effects exist within and
among Earth's systems. \\
ESS2.C The roles of water in Earth's surface processes & The planet's
dynamics are greatly influenced by water's unique chemical and physical
properties. \\
ESS2.D Weather and climate & The role of radiation from the Sun and its
interactions with the atmosphere, ocean, and land are the foundation for
the global climate system. \\
ESS3.A Natural resources & Resource availability has guided the
development of human society and use of natural resources has associated
costs, risks, and benefits, including to global climate. \\
LS1.C Organization for matter and energy flow in organisms & Organisms
are constantly breaking down and reorganizing matter. The hydrocarbon
backbones of sugars produced through photosynthesis are used by
organisms to make amino acids and other macromolecules that can be
assembled into proteins or DNA. During cellular respiration, the bonds
of macromolecules and oxygen are broken down to build new products and
transfer energy. \\
LS2.A Interdependent relationships in ecosystems & Ecosystems have
carrying capacities resulting from biotic and abiotic factors. The
fundamental tension between resource availability and organism
populations affects genetic diversity within populations and
biodiversity within ecosystems. \\
LS2.B Cycles of matter and energy transfer in ecosystems &
Photosynthesis captures energy in sunlight and stores it in chemical
bonds of matter. Most organisms rely on cellular respiration to release
energy in these bonds to power life processes. About 90\% of available
energy is lost from one trophic level to the next, resulting in fewer
organisms at higher levels. At each link in an ecosystem, elements are
combined in different ways and matter and energy are conserved.
Photosynthesis, cellular respiration and decomposition are key
components of the global carbon cycle. \\
LS2.C Ecosystem dynamics, functioning, and resilience & If a biological
or physical disturbance to n ecosystem occurs, including one induced by
human activity, the ecosystem may return to its more or less original
state or become a very different ecosystem, depending on the complex
interactions within the ecosystem. The ability of an ecosystem to both
resist and recover from change is a measure of its overall health. \\
PS1.B Chemical reactions & Chemical processes and reaction rates are
understood in terms of collisions of molecules, rearrangement of atoms,
and changes in energy as determined by properties of elements involved.
Knowledge of conservation of atoms with chemical properties and
electrical charges can be used to describe and predict chemical
reactions. Main types of reactions include transfer of electrons (redox)
or hydronium ions (acids/bases). Changes in pressure, concentration, or
temperature affect the balance between forward and backward reaction
rates (equilibrium). Ionic and covalent bonds can be predicted based on
the types of attractive forces between particles. \\
PS2.B Types of interactions & Electrical forces between electrons and
the nucleus of atoms explain chemical patterns. Intermolecular forces
determine atomic composition, molecular geometry and polarity, and,
therefore, structure and properties of substances. The kinetic-molecular
theory describes the behavior of gas in a system. \\
PS3.A and 3.BDefinition andconservation ofenergy andenergy transfer &
The total energy within a physical system is conserved. Energy transfer
within and between systems can be described and predicted in terms of
energy associated with the motion or configuration of particles. \\
& \\
\bottomrule
\end{longtable}

\hypertarget{scientific-inquiry-and-engineering-design-processes}{%
\paragraph{Scientific Inquiry and Engineering Design
Processes}\label{scientific-inquiry-and-engineering-design-processes}}

Additionally, this curriculum addresses and embraces the Scientific
Inquiry process and Engineering Design process outlined in that
standards framework.

\hypertarget{design-principles}{%
\subsubsection{Design Principles}\label{design-principles}}

This course was built with several design principles in mind:

\begin{enumerate}
\def\labelenumi{\arabic{enumi}.}
\tightlist
\item
  \textbf{Interdisciplinarity}. We believe that the scientific concepts
  of this course, as well as the sociological ones, are inherently
  interdisciplinary in nature. Hence, connections between them are
  considered important and necessary.
\item
  \textbf{Hands On}. We believe that adolescents learn best when they
  can engage directly with materials and concepts, and so this course
  has been designed to include lots of hands on experiences.
\item
  \textbf{Autonomy Support}. We believe that adolescents learn best when
  they are encouraged and allowed to engage with their learning with as
  much autonomy as possible, consistent with research into
  \href{https://selfdeterminationtheory.org}{Self Determination Theory.}
\item
  \textbf{UN Sustainable Development Goals (SDG).} This course has been
  designed to engage adolescents deeply in the
  \href{https://www.un.org/en/sustainable-development-goals}{United
  Nations Sustainable Development Goals (SDG)} and to encourage them as
  \href{https://www.earthcitizens.org/}{Earth Citizens}.
\item
  \textbf{Core Science/Math/Literacy Skills}. This course has been
  designed to support the development and application of core science,
  mathematics, and literacy skills.
\end{enumerate}

\hypertarget{topicsunits-of-study}{%
\subsection{Topics/Units of Study}\label{topicsunits-of-study}}

This curriculum includes fivethese topics/units of study.

\begin{longtable}[]{@{}
  >{\raggedright\arraybackslash}p{(\columnwidth - 2\tabcolsep) * \real{0.5000}}
  >{\raggedright\arraybackslash}p{(\columnwidth - 2\tabcolsep) * \real{0.5000}}@{}}
\toprule
\begin{minipage}[b]{\linewidth}\raggedright
Topic/Unit of Study (35 weeks of instruction)
\end{minipage} & \begin{minipage}[b]{\linewidth}\raggedright
Key Ideas/Concepts
\end{minipage} \\
\midrule
\endhead
\emph{Physical Settings for Living Systems}(4-6 weeks) & History of the
Earth Chemistry of Geological and Atmospheric ProcessesWeather/Evolution
of Weather on Earth \\
\emph{Living and Non-Living Systems}(4-6 weeks) & Chemistry of Living
ThingsFlow of energy through living and non-living systemsHomeostasis in
systemsTrophic Levels \\
\emph{Biodiversity}(4-6 weeks) & What is biodiversity?Why is
biodiversity important?What are the threats to biodiversity?What are
some solutions to the threats of biodiversity? \\
\emph{Sustainability/Human Impacts on the Environment}(8 weeks) & The
dynamic relationship between humans and the environment.Deforestation \&
Habitat lossClimate changeSpecies declinePollutionAttempts to Repair
Impacts to the Environment \\
\emph{Final Project}(8 weeks) & Deconstruct a
\href{https://en.wikipedia.org/wiki/Wicked_problem}{\textbf{``Wicked
Problem''}} related to sustainability.\emph{Research} the problem.
\emph{Propose} a solution. \emph{Design} a prototype. \emph{Involve} the
community. \emph{Share} your work. \\
\bottomrule
\end{longtable}

\hypertarget{course-units}{%
\subsection{Course Units}\label{course-units}}

\hypertarget{unit-1-physical-settings-for-living-systems}{%
\subsubsection{Unit 1: Physical Settings for Living
Systems}\label{unit-1-physical-settings-for-living-systems}}

\hypertarget{unit-2-living-and-non-living-systems}{%
\subsubsection{Unit 2: Living and Non-Living
Systems}\label{unit-2-living-and-non-living-systems}}

\hypertarget{unit-3-biodiversity}{%
\subsubsection{\texorpdfstring{\href{https://docs.google.com/document/d/1Bgrb4qwGBcupsMQ1rD_0cuCFS_clIWAw1MH5NLA0IgM/edit?usp=sharing}{Unit
3: Biodiversity}}{Unit 3: Biodiversity}}\label{unit-3-biodiversity}}

\hypertarget{unit-4-sustainabilityhuman-impacts-on-the-environment}{%
\subsubsection{Unit 4: Sustainability/Human Impacts on the
Environment}\label{unit-4-sustainabilityhuman-impacts-on-the-environment}}

\hypertarget{unit-5-final-project}{%
\subsubsection{Unit 5: Final Project}\label{unit-5-final-project}}

\end{document}
